\subsubsection{Analysis Call and Message Technologies}
\noindent This section below provides how the contact between users works utilising the help of SIP-based service Linphone.
\subsubsubsection{Session Initiation Protocol}

    In order to facilitate data exchange between two parties, 
    establishing a session is essential. However, identifying and connecting with 
    the second participant can be challenging. To address this, specialized protocols 
    have been designed specifically for such scenarios. \\

    \noindent The Session Initiation Protocol \cite{sip}, 
    commonly known as SIP, is a signaling protocol operating at the 
    application layer, designed for Internet telephony, IP-based telephone systems, 
    as well as mobile phone calling over LTE. SIP’s primary purpose is in VoIP systems, where it serves 
    as a support protocol for registering and locating users, and for call set up and management. It can initiate, maintain, 
    and terminate communication sessions that include voice, video and messaging applications. SIP supports five facets of establishing and terminating multimedia communications: 
    user location, user availability, user capabilities, session setup, and session management. \\

    \noindent SIP functions as a standalone application but relies on other protocols 
    to ensure the overall architecture operates effectively. At the transport layer, 
    it is transported by using the Transmission Control Protocol (TCP), the User Datagram Protocol (UDP), 
    or the Transport Layer Security (TLS) depending on specific circumstances. In this project, 
    TLS is used (more on TLS in Section 1.3). \\

    \noindent SIP architecture consists of:
    \begin{itemize}
        \item \textbf {User Agent (UA):} an end point of the network, able to send requests (also known as User Agent Client - UAC) 
        and receive responses (User Agent Server - UAS). User Agent usually acts as both client and server and some examples 
        are - IP phone, softphone, and camera. The caller’s phone acts as a client and the callee’s phone acts as a server.
        \item \textbf {The Proxy Server:} takes a request from a user agent and forwards it to another user (i.e., an INVITE message).
        \item \textbf {The Registrar Server:} which is responsible for registering users to the network. It accepts registration requests from 
        user agents and helps users to authenticate themselves within the network. It stores the URI and the location of users in a database 
        to help other SIP servers within the same domain.
        \item \textbf {The Redirect Server:} receives requests and looks up the intended recipient of the request in the location database created by the registrar. 
        \item \textbf {The Location Server:} provides information regarding the caller’s possible locations to the Redirect and Proxy Servers.
    \end{itemize}
    \noindent In SIP, Each user is identified with a unique address, called SIP Uniform Resource Identifier (SIP URI). It is an address that contains information for establishing a session with the other end. The SIP URI resembles an E-Mail address and is written in the syntax below with the following URI parameters: 
    SIP - URI = sip:x@y:Port  where x = username and y = host (domain or IP).

\subsubsubsection{Linphone}
    Linphone is an open-source VoIP (Voice over Internet Protocol) application that utilizes SIP-based user agent. VoIP messages and calls are made over an IP network rather than over traditional public switched telephone networks (PSTN). 
    Linphone features all basic SIP-related services, such as audio and video calls, call management, call transfer, audio conferencing, and instant messages. 
    
    \pagebreak

    The free Linphone SIP service is released with an open-source license; and the SIP server software powering this service is called Flexisip. 
    Linphone uses its open-source library as its core, called liblinphone. 
    The library is a SIP-based SDK5 for video and audio over IP and is written in C/C++. 
    The application is available on Linux, Windows, MacOS, iOS, and Android. \\

    \noindent The combination of Linphone and Flexisip SIP proxy provides secure end-user registration and call setup.
    More precisely, Linphone client establishes and maintains a SIP TLS connection to the Flexisip server. 
    The Linphone client verifies the SIP server’s identity based on the X.509 digital certificate of the server (a list of trusted root authorities is provided at compilation time). 
    In this way, message and entity authentication, as well as confidentiality, of the information exchanged between the Linphone client and the Flexisip server is ensured. 
\subsubsubsection{Transport Layer Security Protocol (TLS)}
    Transport Layer Security, or TLS, is a widely adopted security protocol designed to facilitate privacy and data security for communications over the Internet. 
    TLS was derived from a security protocol called Secure Socket Layer (SSL). A primary use case of TLS is encrypting the communication between applications and servers. 
    TLS is based on symmetric encryption and is a client - server model, where the client is able to authenticate the server and, optionally, the server is also able to authenticate the client. \\

    \noindent When using SIP over TLS, the whole SIP signalling is encrypted. However it holds only on the segments of the communication which actually use TLS. 
    Therefore, TLS will be automatically set to each end of the user by default in this project. 

