\documentclass{article}

\usepackage{amsmath}

\title{Group Project Report}
\author{}
\date{31/12/2024}

\begin{document}

\begin{abstract}
    this is the abstract for  the group project
\end{abstract}

\section{Introduction}

\subsection{Context and Motivation}

\subsection{Objectives}

    The main goals for this project is to build a mobile application
    that can help to control our university systems, including students'
    grades, study schedule, and especially make an additional feature to connect 
    students who share the same interest of subjects, hobbies so that they can communicate
    and study together after classes.


\subsection{Related works}

In this part we will cite some related works/papers that we used mainly for this 
project. We also summarize the content of these resources.


\section{Background}
\subsubsection{Mobile Application Background}
In this part we will introduce about the mobile application process.


\subsubsection{Clustering Algorithm}
In this part we will explain the theory and mathematical bases for
the clustering algorithm that we used in this project.
\begin{itemize} 
    \item Data encoding method: One Hot Encoding, Word Embedding TD-IDF
    \item Dimensionality reduction
    \item Clustering algorithm: K-mode

\end{itemize}

\subsubsection{Evaluation Framework}
In this part we will explain the evaluation metrics that we used to evaluate our model.
\begin{itemize}
    \item Silhouette score
    \item Davies-Bouldin index

\end{itemize}

 


\section{Material and Methodology}
\subsection{Material}
\subsubsection{Data Sources}
In this part we will explain about the process of gathering data
from scratch, by doing survey.

\subsubsection{Experimental Setup}
In this part, we will explain about the worlflow for both 
application and machine learning part.

\subsection{Methodology}

This part should describe details the implementation process
of both mobile development app and machine learning

Structure: Data describe --> Mobile app process --> Machine Learning Model (integrated inside app) --> Demo for each feature.

\subsubsection{Data Preprocessing}

\subsubsection{Mobile App Process}

\subsubsection{Model Configuration and Training}
In this part we will descrbie detailed about the model building,
how we built it, which function to take, .... and the model training 
process.
\subsection{Model Evaluation}
In this part, we will describe the formula for model evaluation,
which attribute/columns we used to evaluate and how we implement it inside
the model.
\section{Results and Discussion}
\subsection{Results}
In this part we will have a table for illustrate the result,
can compare it with a benchmark result if we get it.

We can also have a demo for the app in this part.

\subsection{Discussion}

\section{Conclusion\&Future Work}
\subsection{Conclusion}
\subsection{Future Work}


\end{document}
