\documentclass{article}
\usepackage{import}
\usepackage{amsmath}
\usepackage[utf8]{inputenc}
\usepackage{graphicx}
\usepackage{setspace}
\usepackage{geometry}
\geometry{a4paper, portrait, margin=30mm, bmargin=30mm, tmargin=30mm}
\setcounter{secnumdepth}{5}
\setcounter{tocdepth}{5}

\makeatletter
\newcommand\subsubsubsection{\@startsection{paragraph}{4}{\z@}{-2.5ex\@plus -1ex \@minus -.25ex}{1.25ex \@plus .25ex}{\normalfont\normalsize\bfseries}}
\newcommand\subsubsubsubsection{\@startsection{subparagraph}{5}{\z@}{-2.5ex\@plus -1ex \@minus -.25ex}{1.25ex \@plus .25ex}{\normalfont\normalsize\bfseries}}
\makeatother


\begin{document}
\setstretch{1.5}
%\vspace{10mm}  % vertical space
\thispagestyle{empty}
% \addvspace{5mm}  % vertical space until length

%$$$$$$$$$$$$$$$$$$$$$$$$$$$$$$$$$$$$$$$$$$$$$$$$$$$$$$$$$$$$$$$$$$$$$$$$$$$$$$$$$

\newgeometry{top=0.6in, bottom=0.6in, left=0.6in, right=0.6in}
% Make the title page
\begin{titlepage}
    \begingroup % Start a local scope
    \renewcommand{\baselinestretch}{1.75}\normalsize 
    \begin{center}
        % Draw a border around the entire page
        \setlength{\fboxrule}{1pt} 
        \setlength{\fboxsep}{2pt} 
        \fbox{
            \begin{minipage}[c][0.98\textheight][c]{0.98\textwidth} % Adjust content width
                \centering
                \vspace{-30mm} % Reduce vertical space at the top
                % University name and department
                \makebox[\textwidth][c]{\normalsize \textbf {UNIVERSITY OF SCIENCE AND TECHNOLOGY OF HANOI}} \\[5pt]
                \makebox[\textwidth][c]{\normalsize \textbf {DEPARTMENT OF INFORMATION AND COMMUNICATION TECHNOLOGY}} \\[20pt]
            
                % USTH logo
                \includegraphics[width=0.5\textwidth]{image/usth.png} \\[15pt]
            
                % Report title
                \textbf{\LARGE GROUP PROJECT REPORT} \\[3pt]
                \textbf{\fontsize{19}{22}\selectfont USTH Connect}\\
                \textbf{\fontsize{19}{22}\selectfont Integrated app for university life assistant \\ and student networking} \\[30pt]
            
                % Group members section
                \textbf{\large Group Members} \\[10pt]
                \begin{tabular}{l@{\hskip 3cm}l} % Add spacing of 3cm between columns
                Nguyen Thi Van    & 22BI13459 \\
                Chu Hoang Viet     & 22BI13462 \\
                Nguyen Hoai Anh        & 22BI13021 \\
                Nguyen Dang Nguyen & 22BI13340 \\
                Do Minh Quang           & 22BI13379 \\
                \end{tabular} \\[20pt]
            
                % Supervisor section
                \textbf{\large Supervisor} \\[3pt]
                Assoc. Prof. Tran Giang Son \\[20pt]
            
                % Footer
                \textbf{January, 2025}
            \end{minipage}
        }
    \end{center}
    \endgroup % End local scope
\end{titlepage}

% Restore margins for the rest of the document
\restoregeometry

\thispagestyle{empty}

% end of title page
%$$$$$$$$$$$$$$$$$$$$$$$$$$$$$$$$$$$$$$$$$$$$$$$$$$$$$$$$$$$$$$$$$$$$$$$$$$$$$$$$$
\newpage

\tableofcontents
\newpage
\begin{abstract}
    this is the abstract for  the group project
\end{abstract}


\section{Introduction}

\subsection{Context and Motivation}
\begin{itemize}
    \item We can ask GPT for this part =))
\end{itemize}
\subsection{Objectives}

    The main goals for this project is to build a mobile application
    that can help to control our university systems, including students'
    grades, study schedule, and especially make an additional feature to connect 
    students who share the same interest of subjects, hobbies so that they can communicate
    and study together after classes.


\subsection{Related works}

In this part we will cite some related works/papers that we used mainly for this 
project. We also summarize the content of these resources.


\section{Background}
\subsubsection{Mobile Application Background}
In this part we will introduce about the standard mobile application framework.


\subsubsection{Machine Learning Background}
In this part we will explain the theory and mathematical bases for
the clustering algorithm that we used in this project.
\begin{itemize} 
    \item Machine Learning Workflow: describe a standard workflow for a clustering
    algorithm.
    \item Data encoding method: One Hot Encoding, Word Embedding TD-IDF
    \item Dimensionality reduction
    \item Clustering algorithm: K-mode (? can we use more algorithm for this part)
    \item Evaluation Metrics: Silhouette score, Davies-Bouldin index

\end{itemize}

\section{Material and Methodology}
\subsection{Material}
\subsubsection{Data Sources}
In this part we will explain about the process of gathering data
from scratch, by doing survey.

\subsubsection{Experimental Setup}
Still consider what to write for this part. Maybe unecessary.

\subsection{Methodology}

This part should describe details the implementation process
of both mobile development app and machine learning

Structure: Data describe $\rightarrow$ Mobile app framework $\leftarrow$ Machine Learning Workflow (integrated inside app) $\rightarrow$ Demo for each feature.

\subsubsection{Mobile App Framework}

\subsubsection{Machine Learning Workflow}
In this part, we will describe more detail about:
\subsubsubsection{Data Preprocessing}

\begin{itemize}
    \item describe the data structure
    \item how and why we using encoding method
    \item how we split data for training and testing
\end{itemize}
\subsubsubsection{Model Configuration and Training}
In this part we will descrbie detailed about 
\begin{itemize}
    \item Model configuration: describe models that we used and why we choose it.
    \item Model training: process of training the model
\end{itemize}

\subsubsubsection{Model Evaluation}
In this part, we will describe:
\begin{itemize}
    \item Attribute we choose to evaluate model performance
    \item Evaluation metrics: Silhouette score, Davies-Bouldin index
\end{itemize}

\section{Results and Discussion}
\subsection{Results}

\subsubsection{Mobile App Results}
In this part we can have the demo for each feature of the app.

\subsubsection{Machine Learning Results}
In this part we will show the result of the clustering algorithm, using the evaluation 
metrics that we mentioned in the previous section.

\subsection{Discussion}

\section{Conclusion {\&} Future Work}

\subsection{Conclusion}
\subsection{Future Work}


\end{document}

